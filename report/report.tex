\documentclass[a4paper, 12pt, conference, portuguese]{ieeeconf}

\IEEEoverridecommandlockouts
\overrideIEEEmargins
\usepackage{graphics}
\usepackage{epsfig}
\usepackage{mathptmx}
\usepackage{times}
\usepackage{amsmath}
\usepackage{amsthm}
\usepackage{amssymb}
\usepackage[left=3cm,right=3cm,top=3cm,bottom=3cm]{geometry}
\usepackage[utf8]{inputenc}
\usepackage[portuguese]{babel}

\usepackage{alltt}
\title{{\LARGE \bf Corte Mínimo de Grafos Pesados\\}
\large{Relatório do 2º Projecto - Análise e Síntese de Algoritmos} }

\author{Baltasar Dinis, 89416 e Afonso Ribeiro, 86752}

\newcommand\comment[1]{}

\comment{
  O relatório deverá ser entregue no formato PDF com não mais de 4 páginas, fonte de 12pt,
  e 3cm de margem. O relatório deverá incluir uma introdução breve, a descrição da solução, a
  análise teórica e a avaliação experimental dos resultados. O relatório deverá incluir qualquer
  referência que tenha sido utilizada na realização do projecto. O texto do relatório deve ser
  cuidado. Serão aplicados descontos por erros gramaticais ou erros ortográficos.
}

\begin{document}
\newtheorem{theorem}{Teorema}
\theoremstyle{definition}
\newtheorem{axiom}{Axioma}[section]
\newtheorem{corollary}{Corolário}
\newtheorem{lemma}{Lema}
\newtheorem{defin}{Definição}
\newtheorem*{dem}{Demonstração}

\maketitle
\thispagestyle{empty}
\pagestyle{empty}

\begin{abstract}
  Redes de distribuição devem ser desenhadas de forma a
  suportar o tráfego a que são submetidas. No entanto, a
  elevada densidade de interconexões entre os produtores,
  centros de distribuição e o destino final torna a avaliação das
  mesmas não óbvia. Neste relatório apresentamos uma solução
  para este problema, permitindo avaliar a capacidade da rede e
  que estações de abastecimento e ligações devem ser aumentadas
  para aumentar essa mesma capacidade. Este trabalho foi
  realizado no contexto da Unidade Curricular de Análise e Síntese
  de Algoritmos, no ano lectivo de 2018-2019.
\end{abstract}

\section{INTRODUÇÃO}\label{intro}

Consideramos que há $3$ categorias de vértices: 1) Os produtores,
que têm um valor de produção $p_i$ associado; 2) As
estações de abastecimento, com capacidade para tratar
uma determinada quantidade de bens; 3) uma estação de destino.
Adicionalmente, cada ligação entre vértices têm um valor máximo
que conseguem suportar.

O objetivo é calcular a capacidade da rede, e obter o fluxo
máximo $F$ de mercadorias, dos produtores para a estação de
destino. Se $F < \sum_i p_i$, então a rede não é adequada, sendo
necessário aumentar a capacidade das estações de abastecimento,
bem como a capacidade das ligações.

O relatório está estruturado da seguinte forma: em \ref{sol} é
apresentada a modelação do problema, o algoritmo e possíveis
ptimizações; em \ref{theoric} é feita uma análise da
complexidade da solução; em \label{experimental} avalia-se
experimentalmente a solução.

\section{DESCRIÇÃO DA SOLUÇÃO}\label{sol}

\subsection{Modelação do Problema}

Representamos o problema com um grafo dirigido pesado, no qual
calculamos o corte mínimo. Consideramos os produtores como
vértices, existindo um nó fantasma, que funciona como fonte, que
se liga aos mesmos. A aresta da fonte $s$ para o produtor $i$ tem
peso $p_i$. Assim conseguimos simular o produtor. Cada estação de
abastecimento expande-se em dois vértices, ligados com uma aresta
cuja capacidade é a da estação. O sentido da aresta é dos produtores
para o destino.

Todas as ligações que precisam de ser aumentadas traduzem-se em arestas
saturadas(pois limitam o fluxo). Só procuramos as estações e
caminhos a aumentar mais próximas do sumidouro, precisamos de calcular
o corte minimo mais próximo do mesmo. Como pretendos fazê-lo a partir
das alturas obtidas pelo algoritmo Push Relabel que da o corte minimo
mais próximo da fonte, è preciso executá-lo na rede transposta.

\subsection{Cálculo do Corte Mínimo}
Calculamos o corte mínimos de acordo com o método
\textit{Push-Relabel}\cite{pre-flow}, escolhendo os vértices com
uma \textit{FIFO}. Saturamos as arestas que
partem da fonte e, enquanto os vértices estiverem ativos, é
aplicada a operação de \textit{discharge}.

Um vértice está ativo se tem excesso, sendo que quando lhe é
inserido excesso (através de uma operação de \textit{push}), é adicionado
a uma fila. A operação de \textit{discharge}, feita a partir do vértice
no topo da fila. Envia todo o fluxo que consegue para os nós que
têm a altura uma unidade mais baixa da sua. Se, depois de fazer
isto, continua com excesso, é feita uma operação de
\textit{relabel}, que aumenta a sua altura.

\subsection{Optimizações}

Há diversas optimizações que podem ser aplicadas, apresentamos
aqui algumas: 1) inicialização de alturas mínimas; 2) cálculo
periódico de alturas mínimas; 3) heurística de espaços .
O objetivo é comum: diminuir o número de operações relabel desnecessárias.

Na inicialização pode ser feita uma procura em largura
a partir do sumidouro no grafo transposto. Isto permite que se inicialize
as alturas com as distâncias da procura, porque o fluxo tem de
chegar à fonte, e para tal acontecer os nós de um nível da árvore
de procura têm de estar mais baixos que os do nível abaixo
(porque estão no caminho mais curto).

Pode ser refeita esta procura em profundidade, novamente a partir
do sumidouro, mas desta vez no transposto do grafo residual, para recalcular as
alturas mínimas necessárias. Como a procura em
profundidade é $O(V + E)$, esta optimização seria feita com
regularidade, mas não em demasia. O valor óptimo pode ser
ajustado experimentalmente.

A terceira optimização é a heurística de espaços. Quando há um
espaço nas alturas dos vértices --- um valor $g < \mid V \mid:
\nexists u \in V : h[u] = g$ --- então podemos elevar todos os
elementos com altura superior ao espaço para $\mid V \mid + 1$,
pois estes vértices não conseguem  atingir o
sumidouro.\cite{goldberg}

Na nossa implementação, optámos por não empregar estas
heurísticas, pois não sentimos que fosse necessário e verificamos
que a criação do grafo era um problema maior para a performance,
devido ao uso da unidade de gestão de memória.

Há outras formas de calcular o fluxo máximo, que potencialmente seriam mais óptimas ---
nomeadamente utilizando a estrutura de árvores dinâmicas
descrita em \cite{pre-flow}, que não consideramos para a
implementação, em nome de evitar optimizações prematuras.

\section{ANÁLISE TEÓRICA}\label{theoric}

Nesta secção fazemos uma breve análise da complexidade do
algoritmo. Assumimos que o leitor está familiarizado com as
operações de \textit{push} e \textit{relabel}, descritas em
\cite{pre-flow}.

\begin{defin} [Push não-saturante]
  Um push diz-se não saturante quando não satura a aresta.
\end{defin}

\begin{lemma} Uma operação de discharge tem no máximo um push não
  saturante. A prova é óbvia: se é feito um push não saturante
  não há mais excesso no vértice para enviar pela areta.
\end{lemma}

\begin{defin} [Passagem]
  Operações de discharge aplicadas nos vértices adicionados na
  passagem anterior.
\end{defin}

\begin{lemma}
  Uma passagem custa $O(V)$, pois é feito no máximo um push não
  saturante por vértice durante a mesma.
\end{lemma}

\begin{lemma} São feitas $4n^2$ passagens na fila \end{lemma}

\begin{dem}
  Seja $H$ a altura máxima dos vértices na fila. Quando é feita
  uma passagem podem acontecer $3$ coisas: 1) $H$
  decresce, o que significa que um nó com altura máxima saiu da
  fila; 2) $H$ mantém-se, o que significa que pelo menos um
  vértice (que tinha $h < H$) aumentou de altura; 3) $H$ aumenta,
  o que significa que um vértice aumentou de altura, tal que
  $\Delta h[v] \geq \Delta H$. Como a altura máxima dos vértices
  é de $2 \mid V \mid < 1$, então as passagens onde $H$ aumenta
  ou mantém-se são $O(V^2)$ (cada vértice só pode aumentar $O(V)$
  vezes). Como $H_0 = 0$ e $H_f = 0$, então as passagens onde $H$
  diminui são também $O(V^2)$.
\end{dem}

\begin{theorem}
  O algoritmo tem complexidade $O(V^3)$.
\end{theorem}
\begin{dem}
  Corolário dos Lemas 2 e 3.
\end{dem}
\section{AVALIAÇÃO EXPERIMENTAL}\label{experimental}

\section{CONCLUSÃO}\label{conclusion}


\begin{thebibliography}{99}
  \bibitem{cormen} T. Cormen, C. Leiserson e L. R. Rivest, \textit{Introduction
    to Algorithms} 1ª edição. Cambridge, Massachussets: The MIT Press,
    1990.
  \bibitem{pre-flow}  A V Goldberg and R E Tarjan. 1986.
    ``A new approach to the maximum flow problem.''
    In Proceedings of the eighteenth annual ACM symposium on Theory of computing (STOC '86). ACM
  \bibitem{goldberg} B V Cherkassky and A V Goldberg. 1994.
    ``On implementing the push-relabel method for the maximum
    flow problem.'' Integer Programming and Combinatorial
    Optimization. IPCO
\end{thebibliography}


\end{document}
