\documentclass[a4paper, 12pt, conference, portuguese]{ieeeconf}

\IEEEoverridecommandlockouts
\overrideIEEEmargins
\usepackage{graphics}
\usepackage{epsfig}
\usepackage{mathptmx}
\usepackage{times}
\usepackage{amsmath}
\usepackage{amssymb}
\usepackage[left=3cm,right=3cm,top=3cm,bottom=3cm]{geometry}
\usepackage[utf8]{inputenc}
\usepackage[portuguese]{babel}

\title{{\LARGE \bf Corte Mínimo de Grafos Pesados\\}
\large{Relatório do 2º Projecto - Análise e Síntese de Algoritmos} }

\author{Baltasar Dinis, 89416 e Afonso Ribeiro, 86752}

\newcommand\comment[1]{}

\comment{
  O relatório deverá ser entregue no formato PDF com não mais de 4 páginas, fonte de 12pt,
  e 3cm de margem. O relatório deverá incluir uma introdução breve, a descrição da solução, a
  análise teórica e a avaliação experimental dos resultados. O relatório deverá incluir qualquer
  referência que tenha sido utilizada na realização do projecto. O texto do relatório deve ser
  cuidado. Serão aplicados descontos por erros gramaticais ou erros ortográficos.
}

\begin{document}
\maketitle
\thispagestyle{empty}
\pagestyle{empty}

\begin{abstract}
  Redes de distribuição devem ser desenhadas de forma a
  suportar o tráfego a que são submetidas. No entanto, a
  elevada densidade de interconexões entre os produtores,
  centros de distribuição e o destino final torna a avaliação das
  mesmas não óbvia. Neste relatório apresentamos uma solução
  para este problema, permitindo avaliar a capacidade da rede e
  que estações de abastecimento e ligações devem ser aumentadas
  para aumentar essa mesma capacidade. Este trabalho foi
  realizado no contexto da Unidade Curricular de Análise e Síntese
  de Algoritmos, no ano lectivo de 2018-2019.
\end{abstract}

\section{INTRODUÇÃO}\label{intro}

Consideramos que há $3$ categorias de vértices: 1) Os produtores,
que têm um valor de produção $p_i$ associado; 2) As
estações de abastecimento, com capacidade para tratar
uma determinada quantidade de bens; 3) uma estação de destino.
Adicionalmente, cada ligação entre vértices têm um valor máximo
que conseguem suportar.

O objetivo é calcular a capacidade da rede, e obter o fluxo
máximo $F$ de mercadorias, dos produtores para a estação de
destino. Se $F < \sum_i p_i$, então a rede não é adequada, sendo
necessário aumentar a capacidade das estações de abastecimento,
bem como a capacidade das ligações.

O relatório está estruturado da seguinte forma: em \ref{sol} é
apresentada a modelação do problema, o algoritmo e possíveis
optimizações; em \ref{theoric} é feita uma análise da
complexidade da solução; em \label{experimental} avalia-se
experimentalmente a solução.

\section{DESCRIÇÃO DA SOLUÇÃO}\label{sol}

\subsection{Modelação do Problema}

Representamos o problema com um grafo dirigido pesado, no qual
calculamos o corte mínimo. Consideramos os produtores como
vértices, existindo um nó fantasma, que funciona como fonte, que
se liga aos mesmos. A aresta da fonte $s$ para o produtor $i$ tem
peso $p_i$. Assim conseguimos simular o produtor. Cada estação de
abastecimento expande-se em dois vértices, ligados com uma aresta
cuja capacidade é a da estação. O sentido da aresta é dos produtores
para o destino.

TODO: explicar porque é que o aumento se encontra no corte.

Como procuramos as estações e caminhos a aumentar mais próximas
do sumidouro, calculamos o corte mínimo no grafo transposto (os
pesos mantêm-se).

\subsection{Cálculo do Corte Mínimo}
Calculamos o corte mínimos de acordo com o método
\textit{Push-Relabel}\cite{pre-flow}, escolhendo os vértices com
uma \textit{FIFO}. Saturamos as arestas que
partem da fonte e, enquanto os vértices estiverem ativos, é
aplicada a operação de \textit{discharge}.

Um vértice está ativo se tem excesso, sendo que quando lhe é
inserido excesso (através de uma operação de \textit{push}), é adicionado
a uma fila. A operação de \textit{discharge}, feita a partir do vértice
no topo da fila. Envia todo o fluxo que consegue para os nós que
têm a altura uma unidade mais baixa da sua. Se, depois de fazer
isto, continua com excesso, é feita uma operação de
\textit{relabel}, que aumenta a sua altura.

\subsection{Optimizações}

Há diversas optimizações que podem ser aplicadas, apresentamos
aqui algumas.

Na inicialização pode ser feita uma procura em lar
no início, uma procura em largura. Isto permite que

\section{AVALIAÇÃO EXPERIMENTAL}\label{experimental}

\section{CONCLUSÃO}\label{conclusion}


\begin{thebibliography}{99}
  \bibitem{cormen} T. Cormen, C. Leiserson e L. R. Rivest, \textit{Introduction
    to Algorithms} 1ª edição. Cambridge, Massachussets: The MIT Press,
    1990.
  \bibitem{pre-flow}  A V Goldberg and R E Tarjan. 1986.
    ``A new approach to the maximum flow problem.''
    In Proceedings of the eighteenth annual ACM symposium on Theory of computing (STOC '86). ACM
\end{thebibliography}
\end{document}
